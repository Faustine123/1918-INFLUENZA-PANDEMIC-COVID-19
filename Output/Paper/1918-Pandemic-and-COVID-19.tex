% Options for packages loaded elsewhere
\PassOptionsToPackage{unicode}{hyperref}
\PassOptionsToPackage{hyphens}{url}
\PassOptionsToPackage{dvipsnames,svgnames,x11names}{xcolor}
%
\documentclass[
]{article}

\usepackage{amsmath,amssymb}
\usepackage{lmodern}
\usepackage{iftex}
\ifPDFTeX
  \usepackage[T1]{fontenc}
  \usepackage[utf8]{inputenc}
  \usepackage{textcomp} % provide euro and other symbols
\else % if luatex or xetex
  \usepackage{unicode-math}
  \defaultfontfeatures{Scale=MatchLowercase}
  \defaultfontfeatures[\rmfamily]{Ligatures=TeX,Scale=1}
\fi
% Use upquote if available, for straight quotes in verbatim environments
\IfFileExists{upquote.sty}{\usepackage{upquote}}{}
\IfFileExists{microtype.sty}{% use microtype if available
  \usepackage[]{microtype}
  \UseMicrotypeSet[protrusion]{basicmath} % disable protrusion for tt fonts
}{}
\makeatletter
\@ifundefined{KOMAClassName}{% if non-KOMA class
  \IfFileExists{parskip.sty}{%
    \usepackage{parskip}
  }{% else
    \setlength{\parindent}{0pt}
    \setlength{\parskip}{6pt plus 2pt minus 1pt}}
}{% if KOMA class
  \KOMAoptions{parskip=half}}
\makeatother
\usepackage{xcolor}
\usepackage[margin = 1in]{geometry}
\setlength{\emergencystretch}{3em} % prevent overfull lines
\setcounter{secnumdepth}{-\maxdimen} % remove section numbering
% Make \paragraph and \subparagraph free-standing
\ifx\paragraph\undefined\else
  \let\oldparagraph\paragraph
  \renewcommand{\paragraph}[1]{\oldparagraph{#1}\mbox{}}
\fi
\ifx\subparagraph\undefined\else
  \let\oldsubparagraph\subparagraph
  \renewcommand{\subparagraph}[1]{\oldsubparagraph{#1}\mbox{}}
\fi


\providecommand{\tightlist}{%
  \setlength{\itemsep}{0pt}\setlength{\parskip}{0pt}}\usepackage{longtable,booktabs,array}
\usepackage{calc} % for calculating minipage widths
% Correct order of tables after \paragraph or \subparagraph
\usepackage{etoolbox}
\makeatletter
\patchcmd\longtable{\par}{\if@noskipsec\mbox{}\fi\par}{}{}
\makeatother
% Allow footnotes in longtable head/foot
\IfFileExists{footnotehyper.sty}{\usepackage{footnotehyper}}{\usepackage{footnote}}
\makesavenoteenv{longtable}
\usepackage{graphicx}
\makeatletter
\def\maxwidth{\ifdim\Gin@nat@width>\linewidth\linewidth\else\Gin@nat@width\fi}
\def\maxheight{\ifdim\Gin@nat@height>\textheight\textheight\else\Gin@nat@height\fi}
\makeatother
% Scale images if necessary, so that they will not overflow the page
% margins by default, and it is still possible to overwrite the defaults
% using explicit options in \includegraphics[width, height, ...]{}
\setkeys{Gin}{width=\maxwidth,height=\maxheight,keepaspectratio}
% Set default figure placement to htbp
\makeatletter
\def\fps@figure{htbp}
\makeatother

\makeatletter
\makeatother
\makeatletter
\makeatother
\makeatletter
\@ifpackageloaded{caption}{}{\usepackage{caption}}
\AtBeginDocument{%
\ifdefined\contentsname
  \renewcommand*\contentsname{Table of contents}
\else
  \newcommand\contentsname{Table of contents}
\fi
\ifdefined\listfigurename
  \renewcommand*\listfigurename{List of Figures}
\else
  \newcommand\listfigurename{List of Figures}
\fi
\ifdefined\listtablename
  \renewcommand*\listtablename{List of Tables}
\else
  \newcommand\listtablename{List of Tables}
\fi
\ifdefined\figurename
  \renewcommand*\figurename{Figure}
\else
  \newcommand\figurename{Figure}
\fi
\ifdefined\tablename
  \renewcommand*\tablename{Table}
\else
  \newcommand\tablename{Table}
\fi
}
\@ifpackageloaded{float}{}{\usepackage{float}}
\floatstyle{ruled}
\@ifundefined{c@chapter}{\newfloat{codelisting}{h}{lop}}{\newfloat{codelisting}{h}{lop}[chapter]}
\floatname{codelisting}{Listing}
\newcommand*\listoflistings{\listof{codelisting}{List of Listings}}
\makeatother
\makeatletter
\@ifpackageloaded{caption}{}{\usepackage{caption}}
\@ifpackageloaded{subcaption}{}{\usepackage{subcaption}}
\makeatother
\makeatletter
\@ifpackageloaded{tcolorbox}{}{\usepackage[many]{tcolorbox}}
\makeatother
\makeatletter
\@ifundefined{shadecolor}{\definecolor{shadecolor}{rgb}{.97, .97, .97}}
\makeatother
\makeatletter
\makeatother
\ifLuaTeX
  \usepackage{selnolig}  % disable illegal ligatures
\fi
\IfFileExists{bookmark.sty}{\usepackage{bookmark}}{\usepackage{hyperref}}
\IfFileExists{xurl.sty}{\usepackage{xurl}}{} % add URL line breaks if available
\urlstyle{same} % disable monospaced font for URLs
\hypersetup{
  pdftitle={Basing Responses to Covid-19 Pandemic from the 1918 Influenza Data should be Avoided because of Their Massive Differences },
  pdfauthor={Randall Ni, Myra Li, Faustine Fan},
  colorlinks=true,
  linkcolor={blue},
  filecolor={Maroon},
  citecolor={Blue},
  urlcolor={Blue},
  pdfcreator={LaTeX via pandoc}}

\title{Basing Responses to Covid-19 Pandemic from the 1918 Influenza
Data should be Avoided because of Their Massive Differences \footnote{Code
  and data are available
  at:https://github.com/Faustine123/1918-INFLUENZA-PANDEMIC-COVID-19.git.
  The orginial paper can be reached out at
  https://www.aeaweb.org/articles?id=10.1257/jel.20201641.}}
\author{Randall Ni, Myra Li, Faustine Fan}
\date{February 18, 2022}

\begin{document}
\maketitle
\begin{abstract}
The Covid-19 pandemic has continued for almost four years now. It has
impacted many aspects of our lives, such as personal wellbeing, anxiety
and depression due to constant lockdowns, and the closing of many small
businesses. On a country level scale, the pandemic wiped out many aged
population, stagnated global economy, and forced many countries to
restrict or close their borders. Beach, Clay, and Saavedra (2022)
examined the Covid-19 pandemic and compared it to the 1918 Influenza;
they argued that the two cases are quite similar, and we should be able
to find out the pattern of the pandemic from the influenza datasets.
However, they missed the valuable insight that the world is simply not
like 1918 anymore, and comparing these two pandemics without thinking
about the social economic setting can lead to dangerous wishful
thinking. We replicated the result of their paper with respect to
mortality rate between age groups, articles mentioning the pandemic,
life expectancy, and world economics. We then investigated whether those
results are useful, and should be considered when responding to the more
recent Covid-19 pandemic. Unfortunately, when considering today's social
economic setting, those older datasets are relatively fruitless and
misleading; our findings are worth considering in order to implement
better health and other guidlines.
\end{abstract}
\ifdefined\Shaded\renewenvironment{Shaded}{\begin{tcolorbox}[breakable, boxrule=0pt, enhanced, interior hidden, frame hidden, sharp corners, borderline west={3pt}{0pt}{shadecolor}]}{\end{tcolorbox}}\fi

\hypertarget{introduction}{%
\subsection{1. Introduction}\label{introduction}}

We are still amidst the COVID-19 pandemic, and it has continued to wreak
havoc on a global scale. The unpreparedness of governments' healthcare
systems and many nation's less than competent leadership all contributed
to the rapid spread of this disease. Due to COVID-19 virus strand's
unique biological similarity to the H1N1 virus that caused the 1918
Influenza (Spanish Flu), many scholars were trying to use the data
recorded back then to predict possible futures for the coronavirus
disease. Furthermore, they were also examining the past data sets in
order to provide guidance on possible future policies. Beach et. al's
paper, ``The 1918 Influenza Pandemic and Its Lessons for COVID-19''
showcases various data sets for the old Influenza and examines the
similarity between them and COVID-19 data. They have touched on many
aspects of the influenza, such as mortality, fertility, media coverage,
and economic growth {[}@beach{]}. Due to the paper's relatively outdated
publication, our team has determined that many of the authors' opinions
about COVID-19 are not sufficient, their influenza graphs are relatively
useless, and other more recent articles should be considered in order to
determine the best course of action for governments around the world.

Although the Beach et. al's paper examines a wide range of topics, our
team has determined that only four topics are worth considering in
today's context: Excess mortality rate/mortality count, media coverage,
life expectancy, and world economics. In the following paper, we have
reproduced four graphs for those chosen topics from the replication
package. Our team will examine the four topics in three stages. Firstly,
we will have a descriptive analysis of the influenza data. Next, an
extensive interpretation of the results will be carried out. Lastly, we
will draw relevant information from related papers to further discuss
interesting implications and the original data's relevance.

\hypertarget{overview-of-additional-selected-articles}{%
\subsubsection{1.1 Overview of Additional Selected
Articles}\label{overview-of-additional-selected-articles}}

Firstly, despite the biological similarity of the H1N1 and COVID-19
virus, we should gain more insight on the most vulnerable age group
through more recent data sets. Centers for Disease Control and
Prevention (CDC)'s mortality count data can provide us with more
information when comparing to the excess mortality rate data provided by
Beach et. al's paper. Although the influenza data covers 13 countries
while CDC's only covers the U.S, we can still see the general pattern
and apply it on a global scale (CDC XXXX).

Secondly, forms of media have changed drastically since 1918, and they
can easily be accessed by civilians. Beach et. al gave a brief overview
of U.S media coverage rate in the 1918, but they did not have any in
depth analysis; the graph itself does not tell much in terms of the
effectiveness of media. Therefore, newer COVID-19 data about media
coverage should be examined as well. Mach et.al examines media reporting
in the U.S, Canada, and Great Britain and graphed the data in terms of
scientific value and sensationalism (Mach et. al 2021). By delving
deeper into their paper, we can understand more about the importance of
news media in the recent pandemic.

Thirdly, life expectancy has also changed since 1918. People generally
have more access to medicine and their lives are relatively healthier
compared to the living conditions in 1918. For that reason, we also need
newer pandemic data to support the claim that Beach et. al made. They
argued that life expectancy around the world generally drops when a
pandemic appears, and Schöley et. al's paper will be used to further
support their claims with more recent and relevant data (Schöley et.
al).

Lastly, the world economy also took a huge hit when COVID-19 forced
countries to close their borders and restricted global trades. Our team
believes that it is imperative to look at both the macroeconomic changes
and the stock market changes during COVID-19. Although the replicated
paper includes data on the 1918 influenza's stock market changes, we
argue that people should consider the unique effects that COVID-19 has
on a global economy. Therefore, Mazur et. al's paper will be used to
analyze the stock market movements during the pandemic, and Barua's
paper will highlight the impact of COVID-19 on global macroeconomics
(Mazur et. al 2021)(Barua 2021).

\hypertarget{data}{%
\subsection{2. Data}\label{data}}

The original paper has more of a literature review style of writing, and
the authors used various data sets to give a general overview of the
1918 Influenza and its similarities with COVID-19.

\hypertarget{source}{%
\subsubsection{2.1 Source}\label{source}}

Our paper will examine different areas of the original paper, and we
will be using different data sets to achieve our goals. The data that we
used for graphing purposes is pulled from various sources, and a general
list can be found below.

\begin{enumerate}
\def\labelenumi{\arabic{enumi})}
\item
  Median excess mortality rate in 13 countries is from Murray et. al,
  ``Estimation of potential global pandemic influenza mortality on the
  basis of vital registry data from the 1918--20 pandemic: a
  quantitative analysis.''(2006)
\item
  Regional patterns of influenza newspaper coverage data is from the
  ``Chronicling America'' newspaper archive published by the Library of
  Congress.
\item
  Period life expectancy by year in 14 Countries data originates from
  ourworldindata.org.
\item
  Stock market indices data is from NBER Macrohistory database.
\end{enumerate}

\hypertarget{methodology}{%
\subsubsection{2.2 Methodology}\label{methodology}}

This analysis will be performed in R {[}@R{]}, using the dplyr
{[}@dplyr{]}, readxl {[}@readxl{]}, tidyr {[}@tidyr{]}, data.table
{[}@table{]}, lubridate {[}@lube{]}, haven {[}@haven{]}, and tidyverse
{[}@tidy{]} packages. All figures in the report are generated using
ggplot2 {[}@ggplot{]}.

The collection method of our sources varies. For
Figure~\ref{fig-2CoivdDeath}, Figure~\ref{fig-3regionalinfluenza},
Figure~\ref{fig-4lifeexpectancy} mentioned in the sources section above,
our team believes that they are relatively impartial and objective. For
source of Figure~\ref{fig-1mortality}, Murray et. al compiled all
countries with high-quality vital registration data for the 1918-20
pandemic and used these data to calculate excess mortality. This might
cause certain biases and make the data inaccurate when applying it
globally. For example, proper data collection just began in 1918, and
only countries that have the resources to properly record the data are
chosen. It is reasonable to argue that they are already developed
nations, and only including them means that the data is not representing
other affected developing nations at the time.

\hypertarget{results}{%
\subsection{3. Results}\label{results}}

Excess mortality rate is the rate of deaths from all causes during a
crisis (in this case, the pandemic) above and beyond what we would have
expected to see under ``normal'' conditions. It is imperative to
understand which group is the most vulnerable, since that information
can aid our healthcare system to mitigate the impact. In figure 1 below,
we have compared the mortality rate with age groups and gender during
the 1918 Influenza. There is a clear spike of excess death rate for male
and female populations in the 25 to 29 age group. In addition, the H1N1
virus seems to be most effective at killing young adults and the middle
age population from 15 to 39. Furthermore, males have a higher chance of
death overall when compared to females during the influenza in those 13
recorded countries.

\begin{figure}

{\centering \includegraphics{1918-Pandemic-and-COVID-19_files/figure-pdf/fig-1mortality-1.pdf}

}

\caption{\label{fig-1mortality}Median Excess Mortality Rate by Age and
Sex in 13 Countries}

\end{figure}

\begin{figure}

{\centering \includegraphics{1918-Pandemic-and-COVID-19_files/figure-pdf/fig-2CoivdDeath-1.pdf}

}

\caption{\label{fig-2CoivdDeath}Covid-19 Death by Age in United States}

\end{figure}

We have produced Figure~\ref{fig-2CoivdDeath} below using the mortality
data set form CDC in order to compare it with the influenza data from
the original paper. Although the CDC data set only includes data
collected from the U.S healthcare system, we believe that the mortality
trend does apply well globally (CDC XXXX). There is a clear uptrend in
mortality when the population is in the older age groups. Furthermore,
the most at risk male population is the 75 to 84 age group, while the
most at risk female population is the 85 years and older age group. In
addition, males are also more likely to die from coronavirus than
females. On the other hand, coronavirus barely affects the younger
population, and we do not see the spike in mortality between teenager to
middle age groups like the influenza data.

Figure~\ref{fig-3regionalinfluenza} below showcases a general overview
of U.S media coverage about the influenza separated by areas. We can see
that there is a clear spike in article mentions for all four areas
during 1918 and it gradually decreases over the years. In addition, out
of the considered areas, Southern states and Midwestern states mention
the most about the influenza. The authors of the original paper did not
do much analysis on the effect of media during a pandemic, since the U.S
did not get affected too much by the H1N1 virus and the data only
applies to the different states. Our team will be using Mach et. al's
paper to supplement our understanding of the role of media in the
discussion section, since their paper does a more systematic and
detailed analysis.

\begin{figure}

{\centering \includegraphics{1918-Pandemic-and-COVID-19_files/figure-pdf/fig-3regionalinfluenza-1.pdf}

}

\caption{\label{fig-3regionalinfluenza}Regional Patterns of Influenza
Newspaper Coverage}

\end{figure}

Figure~\ref{fig-4lifeexpectancy} highlights the life expectancy changes
before, during, and after the 1918 Influenza. Three main elements have
been highlighted with different colours in the graph: Denmark (least
affected), world average, and Italy (most affected). We can see a
general uptick trend in this graph, and there are clear troughs around
1918. This means that the overwhelming majority of the countries
included in this graph experienced decreased life expectancy during the
Influenza era.

\begin{figure}

{\centering \includegraphics{1918-Pandemic-and-COVID-19_files/figure-pdf/fig-4lifeexpectancy-1.pdf}

}

\caption{\label{fig-4lifeexpectancy}Period Life Expectancy by Year in 14
Countries}

\end{figure}

Figure~\ref{fig-5stockmarket} reveals stock market patterns during the
Influenza separated by either the U.S or London stock exchange. Although
this is not an exact reproduction of the original paper, we can still
spot the overall trend. There is no clear pattern shown, and both stock
markets seem to oscillate. They seemed to be relatively unaffected
during the Influenza and did not show drastic increase or decrease in
prices. In addition, both stock market prices gradually increased over
the years and all ended up with a higher price in 1924 (which is the end
of the chart) than in 1915.

\begin{figure}

{\centering \includegraphics{1918-Pandemic-and-COVID-19_files/figure-pdf/fig-5stockmarket-1.pdf}

}

\caption{\label{fig-5stockmarket}Stock Market Indices}

\end{figure}

\hypertarget{discussion}{%
\subsection{4. Discussion}\label{discussion}}

\hypertarget{influenza-and-covid-19-affects-different-age-groups}{%
\subsubsection{4.1 1918 Influenza and COVID-19 Affects Different Age
Groups}\label{influenza-and-covid-19-affects-different-age-groups}}

We can clearly see that the Influenza and COVID-19 affects vastly
different age groups in the above
section(Figure~\ref{fig-1mortality})(Figure~\ref{fig-2CoivdDeath}).
Influenza has a higher lethality amongst the working population, while
older people are more likely to die from COVID-19 (CDC XXXX). Although
it is scientifically proven that those two viruses are very similar in
potential symptoms and biological structure, we shouldn't prepare our
hospitals in the same way as when we are combating the influenza; they
should be modified and expected to receive a large percentage of older
patients. In addition, future policies and social benefits should
prioritize the older generations, since they get affected the most and
are more at risk than other age groups. Furthermore, it is especially
important to prevent or minimize virus transmission to older
generations, since they already have a weaker immune system and the
virus is more effective on them.

\hypertarget{importance-of-media-coverage}{%
\subsubsection{4.2 Importance of Media
Coverage}\label{importance-of-media-coverage}}

Media coverage is an essential element for staying informed and
preventing the rapid spread of pandemics. Despite the original paper's
authors not delving deeper with the Influenza data
(Figure~\ref{fig-3regionalinfluenza}), our team used Mach et. al's paper
to gain further understanding of the power of media during a pandemic.
Mach et. al have a more detailed method of calculating the effectiveness
of media coverage. They only examined traditional news outlets in three
countries (Canada, U.S, and UK), and they separated the data into
``scientific quality score'' and ``sensationalism score''. These
categorizations are very effective, since having high scientific quality
scores means that the information is accurate and having low
sensationalism scores means that the source did not sacrifice too much
factual information to ``grab eyeballs''.

The authors uncover that newspaper reports have varying degrees of
scientific quality. They generally have a moderate score on scientific
quality, while the right leaning media platforms of each country have
significantly lower score (Mach et. al xxxx). The score is extremely low
for validity, precision, context, and the distinction between opinion
versus facts in some cases. The various differences in quality suggest
that readers from Canada, U.S, and UK will obtain varying degrees of
scientific quality when they read different news outlets. Overall,
Canada, does on average, has better scores in this category compared to
the other two countries; it is reasonable to argue that the higher
scientific quality score contributed to their lower COVID-19 cases and
death counts (Mach et. al xxxx).

Another category examined by Mach et. al is sensationalism. While low
sensationalism often correlates with more factual information, it also
means that readers might be uninterested in reading the articles.
Sensationalism scores are low overall across all news outlets, and the
authors suggest that ideology affiliation might influence how media use
information. They highlight that populist-right leaning media is more
likely to include governments' interpretation of the science and
implemented policies in their articles (Mach et. al XXXX).

Overall, new media plays an essential role in providing information for
the public, explaining government actions, and guiding public
perceptions of the pandemic. Through Mach et. al's paper, we can clearly
see that higher scientific quality should be strongly encouraged in news
outlets. It is also reasonable to increase the sensationalism to a
certain degree without infringing the factual correctness. After all,
the news article should be eye-catching enough to grab the attention of
the readers and provide them with the most up-to-date facts to stay
informed about the pandemic.

\hypertarget{life-expectancy-changes}{%
\subsubsection{4.3 Life Expectancy
Changes}\label{life-expectancy-changes}}

Life expectancy is a measure of current population health, and it has an
inverse relationship with mortality. Beach et. al examined the Influenza
data in the original paper, and they argue that COVID-19 should make the
affected countries have similar dips during the pandemic and bounce back
years after {[}@beach{]}. We examined Schöley et. al's paper to obtain
newer data for COVID-19, and we highlight that COVID-19 does have a
similar effect on life expectancy around the world (Schöley et. al).

Schöley et. al uncover that the pandemic has already induced a mortality
shock in the majority of European countries and the U.S by 2021, and
even some of the best performing countries are struggling to keep up
with their pre-pandemic projections. France, Sweden, Belgium, and
Switzerland managed to recover from their losses in 2020 and restored
their life expectancy back to 2019 level. Interestingly, the
Scandinavian countries (Finland, Norway, and Denmark) also managed to
maintain their life expectancy at a pre-pandemic level. Schöley et. al
argue that this phenomenon is caused due to them delivering vaccines
faster on average, introducing various effective non-pharmaceutical
public health interventions, and their effective health care systems.

\hypertarget{macroeconomic-impact-and-the-march-2020-stock-market-crash}{%
\subsubsection{4.4 Macroeconomic impact and the March 2020 Stock Market
Crash}\label{macroeconomic-impact-and-the-march-2020-stock-market-crash}}

Beach et. al provided us with the stock market data during the Influenza
era, but the world is simply not the same anymore. Globalization and the
ease of transportation makes certain countries heavily depend on each
other for materials. The world economy stagnates when COVID-19 forces
governments to restrict borders, introduce lock down measures, and have
tighter controls on imported and exported goods. By examining the
macroeconomics and the stock market, we can have a deeper understanding
of the unprecedented effects of COVID-19.

Barua developed a potential timeline for a country's economy based on
Chinese data. He argues that we should consider both short and long term
effects of COVID-19. To begin with, countries will experience distortion
in supply chains, interruption of human workers, and demand/supply shock
in the short term. Next, distortion in trade will lead to changes in
essential goods demand/supply, loss of employment, and rise in financial
instability. Lastly, countries will experience reduced economic growth
and a possible recession in the long term (Barua xxxx). This timeline
urges governments to reduce the negative effects with fiscal and
monetary policies. Furthermore, this timeline is supported by anecdotal
evidence and Mazur et. al's paper on the stock market evaluation.

Mazur et. al investigated the US stock market performance during the
crash of March 2020 triggered by COVID-19, and they concluded that
healthcare, food, tech, and natural gas stocks earned high positive
returns, while value in real estate, entertainment, hospitality, and
petroleum decreased dramatically (Mazur et. al xxxx). This discovery
coincides with Barua's claims. Essential goods demand such as healthcare
and food sees rise in value, while non-essential goods such as
entertainment suffers heavy losses. The stock market crash in March 2020
also preludes to the possible repossession claim made by Barua.

Overall, although Mazur et. al only examined the U.S stock exchange, it
is considered the most influential stock market and can easily send
ripples of its effect towards other stock exchanges. Therefore, it is
imperative that governments should help the economy to prevent it from
going into a recession and introduce new policies to rebuild it back to
pre-pandemic levels.

\hypertarget{potential-bias}{%
\subsubsection{4.5 Potential Bias}\label{potential-bias}}

Apart from the possible bias introduced in the methodology section,
there are various biases in our sources. To begin with, the data on the
Influenza has various competing factors that we can not account for. For
example, the excess mortality rate graph includes death count from every
cause. World War I, different illness, and poor living conditions all
contributed to the death of young people during that era. Next, the
scope of the data we used to examine the importance of media coverage is
extremely narrow. It only investigates traditional media (news) for
three countries, and misses out on covering various social media
platforms. This data might be biased towards older generations, since
younger cultures obtain most of their information through those
platforms instead of traditional news. Lastly, the paper we used to
examine stock performances during a pandemic focused heavily on the U.S
stock market. While it is true that it can be considered the most
influential stock market in the world, it can be more informative if we
also get to investigate the impact of COVID-19 on other lesser stock
markets such as the Hong Kong stock exchange (HKEX).

\hypertarget{weakness-of-our-paper}{%
\subsubsection{4.6 Weakness of our Paper}\label{weakness-of-our-paper}}

One of the biggest weaknesses of our paper is the limited data sets
obtained from the original paper. During our replications, we discovered
that most of the data sets have been cleaned and already filtered to
answer certain questions. This makes it extremely hard to manipulate the
data sets in other ways to gain more insights and ask more questions.
Furthermore, due to the ongoing nature of COVID-19, new data sets are
constantly developed. Therefore, newer data can completely render our
analysis obsolete or challenge certain arguments made in this paper.

\hypertarget{possible-next-steps}{%
\subsubsection{4.7 Possible Next Steps}\label{possible-next-steps}}

This study can be improved upon by utilizing more data sets to further
confirm or modify the claims made in the above sections. Understanding a
pandemic does not only include how the virus transmits or how it is
constructed biologically. We should also consider the social economical
context when developing new policies for COVID-19. Lastly, due to the
volatile nature of the virus and its constant mutation, it is best to
examine newer papers with more recent data sets when they are available.

\#\newpage{} \#\# References



\end{document}
